\documentclass{article}
\title{\textbf{Setlist Manager}\\ \large{Official User Manual}}
\date{\vspace{-5ex}}
\usepackage[margin=1in]{geometry}
\usepackage{xcolor}
\usepackage{graphicx}
\graphicspath{{img/}}
\newcommand{\imgbox}[2]{\fcolorbox{darkgray}{white}{\includegraphics[width=#1px]{#2}}}
\newcommand{\cont}[1]{\noindent \fcolorbox{darkgray}{orange}{\begin{minipage}{200px} \section{#1} \end{minipage}}\\}
\newcommand{\scont}[1]{\noindent \fcolorbox{darkgray}{olive}{\begin{minipage}{200px} \subsection{#1} \end{minipage}}\\}

\renewcommand{\familydefault}{\sfdefault}
\begin{document}
\maketitle
\fcolorbox{darkgray}{lightgray}{\begin{minipage}{300px} \tableofcontents \end{minipage}}\\\\\\
\cont{Introduction}

Setlist Manager allows musicians to generate semi-random setlists for live music performance.\\

\cont{Catalog Management}

Select the Catalog tab at the top of the window to manage your song catalog.\\
The song catalog is arranged alphabetically by composer, then title.\\

\scont{Open a Song Catalog}
\begin{enumerate}
\item Go to the Catalog tab
\item Click the Open Catalog button located at the right side of the window.
\item Use the file browser to locate a song catalog file.
\end{enumerate}

\scont{Add/Edit a Song}
\begin{enumerate}
	\item Click the Edit button next to an existing song in your catalog, or Click the Add button on the right side of the window.
	\item Enter the song title
	\item Enter the composer
	\item Enter the length of the song in seconds
	\item Enter the song's key
	\item Enter the song's tempo in beats-per-minute(BPM)
	\item Enter an introduction time in seconds. The introduction time can be used to tell a story, explain a song's meaning, or introduce members of the band.
	\item Enter the song's genre. For songs that fit multiple genres, use a comma to in between each genre.. 
	\item The Archive checkbox prevents a song from being added to a setlist, but it will still be visible in the song catalog.
\end{enumerate}

\scont{Save/Export the Song Catalog}

\cont{Setlist Creation}

\scont{Configure Setlist Length and Breaks}
\scont{Filter Songs}
\scont{Generate Setlist}
Select the Generate Setlist button to create a new setlist based on your settings.\\
You may see a warning, and generate an incomplete setlist if there are not enough songs in the catalog.
\scont{Export/Share a Setlist}
\end{document}
